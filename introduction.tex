\section{Введение}
\subsection{Актуальность и обоснование проблемы}
На текущий момент приобретение специализированных шахмат для слабовидящих людей
затруднено, так как стоимость таких шахмат достаточно высока, чтобы оттолкнуть
человека от их покупки. Использование слабовидящими людьми обычных шахмат
практически невозможно, так как фигуры на обычной доске очень не устойчивы. 

По данным последней статистики~\cite{web:min-educ}, в России слабовидящими и
слепыми людьми признаны более 210'000 человек. Это население среднего
российского города. И многие из этих людей не могут сыграть в шахматы из-за
дороговизны шахматного набора или из-за простого не знания о существовании
специализированных шахмат для слабо зрячих людей. Это актуализирует проблему
высокой стоимости шахмат и малой осведомлённости о  разновидности игры в
шахматы о слепых и организациях, подобных \acrshort{ibca}~\cite{web:wiki-ibca}.\@

\subsection{Тема работы}
Специализированные \gls{chess} для слабовидящих людей

\subsection{Цель работы}
\begin{enumerate}
    \item Проанализировать исторические прототипы и современные аналоги;
    \item Предложить решение выявленных проблем;
    \item Реализовать предложенные методы решения.
\end{enumerate}

\subsection{Задачи}
\begin{enumerate}
    \item Разработать тактильные шахматы для слабо зрячих людей
    \item Разработать метод, позволяющий расширить аудиторию, осведомлённую о
        существовании специализированных шахмат для слабо зрячих людей.
\end{enumerate}
\newpage
