\documentclass{beamer}

\usepackage{tikz}
\usetheme{Berlin}
\usecolortheme{beaver}

\usepackage[main=russian,english]{babel} 

%Титульный лист
\title[тактильные шахматы]{ТАКТИЛЬНЫЕ ШАХМАТЫ}
\subtitle{проектная работа}
\author{Максимов Матвей}
\institute[МБОУ СОШ №73]{%
    Муниципальное бюджетное общеобразовательное учреждение\\
    <<Средняя общеобразовательная школа №73>>}
\date{Февраль 2023 г.}

\begin{document}
\frame{\titlepage}

\section{Введение}

\begin{frame}
    \begin{block}{Тема работы}
        Специализированные шахматы для слабовидящих людей
    \end{block}\pause%

    \begin{block}{Цель работы}
        \begin{enumerate}
            \item Проанализировать исторические прототипы и современные аналоги
            \item Предложить решение выявленных проблем
            \item Реализовать предложенные методы решения
        \end{enumerate}
    \end{block}
\end{frame}

\begin{frame}
    \begin{block}{Задачи}
        \begin{enumerate}
            \item Разработать тактильные шахматы для слабовидящих людей
            \item Разработать метод, позволяющий расширить аудиторию, осведомлённую
                о существовании специализированных шахмат для слабовидящих людей
        \end{enumerate}
    \end{block}
\end{frame}

\section{Теоретический этап}

\begin{frame}
    \frametitle{Теоретический этап}
\end{frame}

\section{Практический этап}

\begin{frame}
    \frametitle{Практический этап}
\end{frame}

\section{Заключение}

\begin{frame}
    \frametitle{Заключение}
\end{frame}
\end{document}
