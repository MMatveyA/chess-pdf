\section{Введение}
На текущий момент приобретение специализированных шахмат для слабовидящих людей
затруднено, так как стоимость таких шахмат достаточно высока, чтобы оттолкнуть
человека от их покупки. Использование слабовидящими людьми обычных шахмат
практически невозможно, так как фигуры на обычной доске очень не устойчивы. 

По данным последней статистики~\cite{web:min-educ}, в России слабовидящими и
слепыми людьми признаны более 210'000 человек. Это население среднего
российского города. И многие из этих людей не могут сыграть в шахматы из-за
высокой стоимости шахматного набора или из-за простого не знания о существовании
специализированных шахмат для слабо зрячих людей. Это актуализирует проблему
малой доступности шахмат для слепых и малой осведомлённости людей о разновидности игры в
шахматы для слепых и организациях, подобных \acrshort{ibca}~\cite{web:wiki-ibca}.\@

\subsection*{Объект проектной работы}
Специализированные \gls{chess} для слабовидящих людей

\subsection*{Цель работы}
\begin{enumerate}
    \item Разработка тактильных шахмат для слабо зрячих людей
    \item Разработка метода, позволяющего расширить аудиторию, осведомлённую о
        существовании специализированных шахмат для слабо зрячих людей.
\end{enumerate}

\subsection*{Задачи}
\begin{enumerate}
    \item Анализ исторических прототипов и современных аналогов;
    \item Предложение решения выявленных проблем;
    \item Реализация предложенных методов решения.
\end{enumerate}

\subsection*{Методы исследования}
\begin{itemize}
    \item изучение и сравнительный анализ исторических и современных аналогов;
    \item анализ материалов по заданной теме из научно--популярной литературы,
        интернет--ресурсов;
\end{itemize}

\subsection*{Новизна}
Использование оригинальных шахматных фигур. Разработка рекламы,
специализированной для интернет--площадок.

\subsection*{Практическая значимость}
Изготовленный шахматный комплект может использоваться для игры в \gls{chess} не
только слабовидящими людьми, но и полностью здоровыми людьми. Так же его
возможно использовать для обучения игре в шахматы и изучения шахматной нотации.

\subsection*{Этапы работы над проектом}
\begin{enumerate}
    \item Выбор темы
    \item Определение целей и задач проектной работы
    \item Подбор и изучение материалов по теме
    \item Составление библиографии
    \item Разработка шахматного набора
    \item Оформление проектной работы
\end{enumerate}
